\documentclass{article}
\usepackage{CJKutf8}
\usepackage{amsmath}
\usepackage{amssymb}
\usepackage{bm}

\begin{document}
\begin{CJK*}{UTF8}{gbsn}

\title{正则方程组系数矩阵}
\date{}
\maketitle

如果向量组$\bm{\varphi}_0, \bm{\varphi}_1, ..., \bm{\varphi}_n$线性无关,则正则方程组的系数矩阵是对称正定矩阵。\\

\textbf{证明}

正则方程组系数矩阵:
\[
A = 
\left(
\begin{array}{cccc}
(\bm{\varphi}_0, \bm{\varphi}_0) & (\bm{\varphi}_0, \bm{\varphi}_1) & \ldots & (\bm{\varphi}_0, \bm{\varphi}_n) \\
(\bm{\varphi}_1, \bm{\varphi}_0) & (\bm{\varphi}_1, \bm{\varphi}_1) & \ldots & (\bm{\varphi}_1, \bm{\varphi}_n) \\
\vdots & \vdots & \ddots & \vdots \\
(\bm{\varphi}_n, \bm{\varphi}_0) & (\bm{\varphi}_n, \bm{\varphi}_1) & \ldots & (\bm{\varphi}_n, \bm{\varphi}_n) \\
\end{array}
\right)
\]

下面对$n$归纳证明$A$对称正定,即$\forall \bm{x} \neq \bm{0}, \bm{x}^\top A \bm{x} > 0$ \\

$\textcircled{1} n = 1, \bm{x}_0(\bm{\varphi}_0, \bm{\varphi}_0)\bm{x}_0 > 0$显然成立 \\

$\textcircled{2}$假设$n = k$时成立,即 \\
\[ \bm{x}_k^\top A_k \bm{x}_k > 0 \]
其中
\[
\bm{x}_k =
\left(
\begin{array}{cccc}
x_1 & x_2 & \ldots & x_k
\end{array}
\right)^\top
\]
\[
A_k = 
\left(
\begin{array}{cccc}
(\bm{\varphi}_0, \bm{\varphi}_0) & (\bm{\varphi}_0, \bm{\varphi}_1) & \ldots & (\bm{\varphi}_0, \bm{\varphi}_k) \\
(\bm{\varphi}_1, \bm{\varphi}_0) & (\bm{\varphi}_1, \bm{\varphi}_1) & \ldots & (\bm{\varphi}_1, \bm{\varphi}_k) \\
\vdots & \vdots & \ddots & \vdots \\
(\bm{\varphi}_k, \bm{\varphi}_0) & (\bm{\varphi}_k, \bm{\varphi}_1) & \ldots & (\bm{\varphi}_k, \bm{\varphi}_k) \\
\end{array}
\right)
\]

$n = k + 1$时 \\
\leftline{$\bm{x}_{k + 1}^\top A_{k + 1} \bm{x}_{k + 1} = $}
\[
\left(
\begin{array}{ccccc}
x_1 & x_2 & \ldots & x_k & x_{k + 1}
\end{array}
\right)
\left(
\begin{array}{ccccc}
(\bm{\varphi}_0, \bm{\varphi}_0) & (\bm{\varphi}_0, \bm{\varphi}_1) & \ldots & (\bm{\varphi}_0, \bm{\varphi}_k) & (\bm{\varphi}_0, \bm{\varphi}_{k + 1}) \\
(\bm{\varphi}_1, \bm{\varphi}_0) & (\bm{\varphi}_1, \bm{\varphi}_1) & \ldots & (\bm{\varphi}_1, \bm{\varphi}_k) & (\bm{\varphi}_1, \bm{\varphi}_{k + 1}) \\
\vdots & \vdots & \ddots & \vdots & \vdots \\
(\bm{\varphi}_k, \bm{\varphi}_0) & (\bm{\varphi}_k, \bm{\varphi}_1) & \ldots & (\bm{\varphi}_k, \bm{\varphi}_k) & (\bm{\varphi}_k, \bm{\varphi}_{k + 1}) \\
(\bm{\varphi}_{k + 1}, \bm{\varphi}_0) & (\bm{\varphi}_{k + 1}, \bm{\varphi}_1) & \ldots & (\bm{\varphi}_{k + 1}, \bm{\varphi}_k) & (\bm{\varphi}_{k + 1}, \bm{\varphi}_{k + 1})
\end{array}
\right)
\left(
\begin{array}{c}
x_1 \\ x_2 \\ \vdots \\ x_k \\ x_{k + 1}
\end{array}
\right)
\]

记
\[
\bm{\alpha}_k =
\left(
\begin{array}{c}
(\bm{\varphi}_0, \bm{\varphi}_{k + 1}) \\
(\bm{\varphi}_1, \bm{\varphi}_{k + 1}) \\
\vdots \\
(\bm{\varphi}_k, \bm{\varphi}_{k + 1})
\end{array}
\right)
\]
\[
\bm{\beta}_k =
((\bm{\varphi}_{k + 1}, \bm{\varphi}_0) \ 
(\bm{\varphi}_{k + 1}, \bm{\varphi}_1) \ 
\ldots \ 
(\bm{\varphi}_{k + 1}, \bm{\varphi}_k))
\]

于是上式可写为
\[
\begin{aligned}
\left(
\begin{array}{cc}
\bm{x}_k^\top & x_{k + 1}
\end{array}
\right)
\left(
\begin{array}{cc}
A_k & \bm{\alpha}_k \\
\bm{\beta}_k & (\bm{\varphi}_{k + 1}, \bm{\varphi}_{k + 1})
\end{array}
\right)
\left(
\begin{array}{c}
\bm{x}_k \\ x_{k + 1}
\end{array}
\right)
\end{aligned}
\]

\[
\begin{aligned}
&=
\left(
\begin{array}{cc}
\bm{x}^\top A_k + x_{k + 1}\bm{\beta}_k & \bm{x}_k^\top \bm{\alpha}_k + x_{k + 1}
\left(
\begin{array}{cc}
\bm{\varphi}_{k + 1} & \bm{\varphi}_{k + 1}
\end{array}
\right)
\end{array}
\right)
\left(
\begin{array}{c}
\bm{x}_k \\
x_{k + 1}
\end{array}
\right) \\
&=
\bm{x}_k^\top A_k \bm{x}_k +
x_{k + 1} \bm{\beta}_k \bm{x}_k +
\bm{x}_k^\top \bm{\alpha}_k x_{k + 1} +
x_{k + 1}
\left(
\begin{array}{cc}
\bm{\varphi}_{k + 1} & \bm{\varphi}_{k + 1}
\end{array}
\right)
x_{k + 1} \\
&=
\bm{x}_k^\top A_k\bm{x}_k + 2x_{k + 1}\bm{\alpha}_k^\top\bm{x}_k + x_{k + 1}
\left(
\begin{array}{cc}
\bm{\varphi}_{k + 1} & \bm{\varphi}_{k + 1}
\end{array}
\right)
\end{aligned}
\]



\end{CJK*}
\end{document}