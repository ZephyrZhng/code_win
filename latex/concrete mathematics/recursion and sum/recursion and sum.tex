\documentclass{article}
\usepackage{CJKutf8}
\usepackage{amsmath}
\usepackage{amssymb}
\usepackage{bm}
\usepackage{graphicx}
\usepackage{wasysym}
\usepackage{mathdots}

\begin{document}
\begin{CJK*}{UTF8}{gbsn}

\newcommand \fallingfactorialpower[2]{{#1}^{\underline{#2}}}
\newcommand \columnvector[2]{
	\left(
	\begin{array}{c}
	{#1}_1 \\ {#1}_2 \\ \vdots \\ {#1}_{#2}
	\end{array}
	\right)
}
\newcommand \rowvector[2]{
	\left(
	\begin{array}{cccc}
	{#1}_1 & {#1}_2 & \ldots & {#1}_{#2}
	\end{array}
	\right)
}

\title{关于$1$到$n$的$k$次方之和的问题}
\author{张艺瀚}
\date{}
\maketitle

定义:
\[
S_k(n) = \sum_{i = 1}^n i^k
\]

\section{方法$1$:使用二项展开}

\[
\begin{aligned}
& 1^{k + 1} = 1 \\
& 2^{k + 1} = (1 + 1)^{k + 1} = \sum_{i = 0}^{k + 1} \binom {k + 1} i 1^i = \binom {k + 1} 0 1^0 + \binom {k + 1} 1 1^1 + \ldots + \binom {k + 1} {k + 1} 1^{k + 1} \\
& 3^{k + 1} = (2 + 1)^{k + 1} = \sum_{i = 0}^{k + 1} \binom {k + 1} i 2^i = \binom {k + 1} 0 2^0 + \binom {k + 1} 1 2^1 + \ldots + \binom {k + 1} {k + 1} 2^{k + 1} \\
& \ldots \\
& (n + 1)^{k + 1} = \sum_{i = 0}^{k + 1} \binom {k + 1} i n^i = \binom {k + 1} 0 n^0 + \binom {k + 1} 1 n^1 + \ldots + \binom {k + 1} {k + 1} n^{k + 1} \\
\end{aligned}
\]

将上面的等式左右相加,除第一个外,每个式子的左边与下个式子的右边最后一项相消:
\[
\begin{aligned}
(n + 1)^{k + 1}	& = 1 + \binom {k + 1} 0 \sum_{i = 1}^n i^0 + \binom {k + 1} 1 \sum_{i = 1}^n i^1 + \ldots + \binom {k + 1} k \sum_{i = 1}^n i^k \\
				& = 1 + \sum_{j = 0}^{k - 1} \left( \binom {k + 1} j \sum_{i = 1}^n i^j \right) + (k + 1)S_k(n) \\
\end{aligned}
\]

从而:
\[
\begin{aligned}
S_k(n)	& = \frac{1}{k + 1} \left[ (n + 1)^k - \sum_{j = 0}^{k - 1} \left( \binom {k + 1} j \sum_{i = 1}^n i^j \right) - 1 \right] \\
		& = \frac{1}{k + 1} \left[ (n + 1)^k - \sum_{j = 0}^{k - 1} \binom {k + 1} j S_j(n) - 1 \right]
\end{aligned}
\]

这是一个很强的递归式,尝试将它化为封闭形式,以失败告终。这个式子对我们没什么帮助。
事实上,有一个阿尔哈曾公式:
\[
S_{k + 1}(n) = (n + 1)S_k^n - \sum_{i = 1}^n \sum_{j = 1}^i j^k
\]

这个公式可以用几何方法很容易的证明。

\section{方法$2$:使用下降阶乘幂}
下降阶乘幂指的是:
\[
\fallingfactorialpower{n}{k} = x(x - 1)\ldots(x - k + 1)
\]

首先看如何将正常的幂转化为下降阶乘幂,即求下面的$a_1, a_2, \ldots a_k$
\[
n^k = a_1\fallingfactorialpower{n}{k} + a_2\fallingfactorialpower{n}{k - 1} + \ldots + a_k\fallingfactorialpower{n}{1}
\]

分别取$n$为$1, 2, \ldots k$,得到下面的关于$a_1, a_2, \ldots a_k$的线性方程组:
\[
\begin{cases}
1^k = a_1\fallingfactorialpower{1}{k} + a_2\fallingfactorialpower{1}{k - 1} + \ldots + a_k\fallingfactorialpower{1}{1} \\
2^k = a_1\fallingfactorialpower{2}{k} + a_2\fallingfactorialpower{2}{k - 1} + \ldots + a_k\fallingfactorialpower{2}{1} \\
\ldots \\
k^k = a_1\fallingfactorialpower{k}{k} + a_2\fallingfactorialpower{k}{k - 1} + \ldots + a_k\fallingfactorialpower{k}{1} \\
\end{cases}
\]

写成喜闻乐见的矩阵形式就是:
\[
\left(
\begin{array}{cccc}
\fallingfactorialpower{1}{k} & \fallingfactorialpower{1}{k - 1} & \ldots & \fallingfactorialpower{1}{1} \\
\fallingfactorialpower{2}{k} & \fallingfactorialpower{2}{k - 1} & \ldots & \fallingfactorialpower{2}{1} \\
\vdots & \vdots & \ddots & \vdots \\
\fallingfactorialpower{k}{k} & \fallingfactorialpower{k}{k - 1} & \ldots & \fallingfactorialpower{k}{1} \\
\end{array}
\right)
\columnvector{a}{k} =
\left(
\begin{array}{c}
1^k \\ 2^k \\ \vdots \\ k^k
\end{array}
\right)
\]

根据$\fallingfactorialpower{x}{k}$的定义我们知道,$x$取$0, 1, \ldots k - 1$时,$\fallingfactorialpower{x}{k}$均为$0$,从而:
\[
\left(
\begin{array}{cccc}
								&			&								&\fallingfactorialpower{1}{1}	\\
								&			&\fallingfactorialpower{2}{2}	&\fallingfactorialpower{2}{1}	\\
								&\iddots	&\vdots							&\vdots							\\
\fallingfactorialpower{k}{k}	&\ldots		&\fallingfactorialpower{k}{2}	&\fallingfactorialpower{k}{1}	\\	
\end{array}
\right)
\columnvector{a}{k} =
\left(
\begin{array}{c}
1^k \\ 2^k \\ \vdots \\ k^k \\
\end{array}
\right)
\]

我们知道:
\[
\fallingfactorialpower{n}{k} =
\begin{cases}
0 & n < k \\
\dfrac{n!}{(n - k)!} = k!  \dbinom{n}{k} & n \geq k
\end{cases}
\]

从而:
\[
\left(
\begin{array}{cccc}
	&			&						&\dfrac{1!}{(1 - 1)!}	\\
	&			&2!						&\dfrac{2!}{(2 - 1)!}	\\
	&\iddots	&\vdots					&\vdots					\\
k!	&\ldots		&\dfrac{k!}{(k - 2)!}	&\dfrac{k!}{(k - 1)!}	\\
\end{array}
\right)
\columnvector{a}{k} =
\left(
\begin{array}{c}
1^k \\ 2^k \\ \vdots \\ k^k \\
\end{array}
\right)
\]

化简得:
\[
\left(
\begin{array}{cccc}
	&			&						&1						\\
	&			&1						&\dfrac{1}{1!}			\\
	&\iddots	&\vdots					&\vdots					\\
1	&\ldots		&\dfrac{1}{(k - 2)!}	&\dfrac{1}{(k - 1)!}	\\
\end{array}
\right)
\columnvector{a}{k} =
\left(
\begin{array}{c}
\dfrac{1^k}{1!} \\ \dfrac{2^k}{2!} \\ \vdots \\ \dfrac{k^k}{k!} \\
\end{array}
\right)
\]

容易看到:
\[
a_j =
\begin{cases}
1 & j = 1 \\
(j - 1)!\left( \dfrac{j^k}{j!} - \dfrac{(j - 1)^k}{(j - 1)!} \right) = j^{k - 1} - (j - 1)^k & j = 2, 3, \ldots k
\end{cases}
\]

这样我们就有:
\[
\begin{aligned}
n^k	& = \sum_{j = 1}^k a_j \fallingfactorialpower{n}{k - j + 1} \\
	& = \sum_{j = 1}^k (j^{k - 1} - (j - 1)^k) \fallingfactorialpower{n}{k - j + 1}
\end{aligned}
\]

于是:
\[
\begin{aligned}
\sum_{0 \leq n < N} n^k	& = 0^k + 1^k + \ldots + (N - 1)^k \\
						& = \sum_{j = 1}^k \left[ (j^{j - 1} - (j - 1)^k) \dfrac{\fallingfactorialpower{N}{k - j + 2}}{k - j + 2} \right]
\end{aligned}
\]
\[
\begin{aligned}
\sum_{0 \leq n < N + 1} n^k	& = \sum_{n = 1}^N n^k \\
							& = \sum_{j = 1}^k \left[ (j^{j - 1} - (j - 1)^k) \dfrac{\fallingfactorialpower{(N + 1)}{k - j + 2}}{k - j + 2} \right]
\end{aligned}
\]

\end{CJK*}
\end{document}