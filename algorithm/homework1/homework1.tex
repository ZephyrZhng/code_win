\documentclass{article}

\usepackage{amsmath}
\usepackage{amssymb}
\usepackage{bm}
\usepackage{color}
\usepackage{CJKutf8}
\usepackage{color}
\usepackage{enumitem}
\usepackage{graphicx}
\usepackage{listings}
\usepackage{mathdots}
\usepackage{tikz}
\usepackage{wasysym}
\usepackage{xcolor}

\usetikzlibrary{shapes,arrows}

\allowdisplaybreaks

\definecolor{mygreen}{rgb}{0,0.6,0}
\definecolor{mygray}{rgb}{0.5,0.5,0.5}
\definecolor{mymauve}{rgb}{0.58,0,0.82}

\lstset{ %
  backgroundcolor=\color{white},   % choose the background color; you must add \usepackage{color} or \usepackage{xcolor}
  basicstyle=\footnotesize,        % the size of the fonts that are used for the code
  breakatwhitespace=false,         % sets if automatic breaks should only happen at whitespace
  breaklines=true,                 % sets automatic line breaking
  captionpos=b,                    % sets the caption-position to bottom
  commentstyle=\color{mygreen},    % comment style
  deletekeywords={...},            % if you want to delete keywords from the given language
  escapeinside={\%*}{*)},          % if you want to add LaTeX within your code
  extendedchars=true,              % lets you use non-ASCII characters; for 8-bits encodings only, does not work with UTF-8
  frame=single,                    % adds a frame around the code
  keepspaces=true,                 % keeps spaces in text, useful for keeping indentation of code (possibly needs columns=flexible)
  keywordstyle=\color{blue},       % keyword style
  language=C++,                    % the language of the code
  morekeywords={*,...},            % if you want to add more keywords to the set
  numbers=left,                    % where to put the line-numbers; possible values are (none, left, right)
  numbersep=5pt,                   % how far the line-numbers are from the code
  numberstyle=\tiny\color{mygray}, % the style that is used for the line-numbers
  rulecolor=\color{black},         % if not set, the frame-color may be changed on line-breaks within not-black text (e.g. comments (green here))
  showspaces=false,                % show spaces everywhere adding particular underscores; it overrides 'showstringspaces'
  showstringspaces=false,          % underline spaces within strings only
  showtabs=false,                  % show tabs within strings adding particular underscores
  stepnumber=1,                    % the step between two line-numbers. If it's 1, each line will be numbered
  stringstyle=\color{mymauve},     % string literal style
  tabsize=2,                       % sets default tabsize to 2 spaces
  title=\lstname                   % show the filename of files included with \lstinputlisting; also try caption instead of title
}

\begin{document}
\begin{CJK*}{UTF8}{gbsn}

\title{Linear Time Selection}
\author{计算机1202 张艺瀚}
\maketitle

\begin{lstlisting}
#include <algorithm>
#include <cassert>
#include <cstdio>
#include <cstdlib>
#include <ctime>
#include <fstream>
#include <functional>
#include <iostream>
#include <iterator>
#include <limits>
#include <map>
#include <numeric>
#include <regex>
#include <set>
#include <string>
#include <vector>

using namespace std;

int cnt = 0;

void disp(const vector<int>& a)
{
	for(auto& p: a)
		cout << p << " ";
	cout << endl;
}

int my_partition(vector<int>& a, int p, int r, int x)
{
	swap(*find(a.begin(), a.end(), x), a[p]);

	int i = p;
	int j = r;
	while(i < j)
	{
		while(i < j && a[j] >= x) --j;
		a[i] = a[j];
		while(i < j && a[i] <= x) ++i;
		a[j] = a[i];
	}
	a[i] = x;
	return i;
}

int select(vector<int> a, int p, int r, int k)
{
	++cnt;
	cout << "call select " << cnt << ":" << endl;

	if(r - p + 1 <= 5)
	{
		cout << "recursion exit:" << endl;
		vector<int> my_a(r - p + 1);
		copy(a.begin() + p, a.begin() + r + 1, my_a.begin());
		cout << "the sub-vector whose length is no greater than 5 is:" << endl;
		disp(my_a);
		sort(my_a.begin(), my_a.end());
		cout << "return: " << my_a[p + k - 1] << endl << endl;

		cout << endl << "left select from recursion exit" << cnt << endl;
		return my_a[p + k - 1];
	}

	cout << "find median of medians:" << endl;
	vector<int> medians;
	for(int i = 0; i < (r - p + 1) / 5; ++i)
	{
		vector<int> my5;
		for(int j = 0; j < 5; ++j)
		{
			my5.push_back(a[p + 5 * i + j]);
		}
		sort(my5.begin(), my5.end());
		medians.push_back(my5[2]);
	}
	cout << "median is in:" << endl;
	disp(medians);
	int x = select(medians, 0, medians.size() - 1, (medians.size() - 1) / 2 + 1);
	cout << "median is: " << x << endl << endl;

	vector<int> my_a(a);
	int i = my_partition(my_a, p, r, x);
	int j = i - p + 1;
	cout << "after partition, a becomes:" << endl;
	disp(my_a);
	cout << endl;

	if(k <= j)
	{
		vector<int> l_a(i - p + 1);
		copy(my_a.begin() + p, my_a.begin() + i + 1, l_a.begin());
		cout << "next, find in left part:" << endl;
		disp(l_a);
		cout << endl;

		cout << endl << "left select from left recursion" << cnt << endl;
		return select(l_a, 0, l_a.size() - 1, k);
	}
	else
	{
		vector<int> r_a(r - i);
		copy(my_a.begin() + i + 1, my_a.begin() + r + 1, r_a.begin());
		cout << "next, find in right part:" << endl;
		disp(r_a);
		cout << endl;

		cout << endl << "left select from right recursion" << cnt << endl;
		return select(r_a, 0, r_a.size() - 1, k - j);
	}
}

int main()
{
	vector<int> a = {
		29,22,28,14,45,
		10,44,23, 9,39,
		38,52, 6, 5,50,
		37,11,26, 3,15,
		 2,53,40,54,25,
		42,12,19,30,16,
		18,13, 1,48,41,
		24,43,46,47,17,
		34,20,31,32,33,
		35, 4,49,51, 7,
		36,27, 8,21
	};

	cout << "a is:" << endl;
	disp(a);
	cout << endl;

	cout << "the 15th least element of a is: " << select(a, 0, a.size() - 1, 15) << endl << endl;

	cout << "sorted a is:" << endl;
	sort(a.begin(), a.end());
	disp(a);
	cout << "the 15th least element of a is: " << a[14] << endl;

	return 0;
}
\end{lstlisting}



\begin{lstlisting}[
  backgroundcolor=\color{white},   % choose the background color; you must add \usepackage{color} or \usepackage{xcolor}
  basicstyle=\ttfamily\small,        % the size of the fonts that are used for the code
  breakatwhitespace=false,         % sets if automatic breaks should only happen at whitespace
  breaklines=true,                 % sets automatic line breaking
  captionpos=b,                    % sets the caption-position to bottom
  commentstyle=\color{mygreen},    % comment style
  deletekeywords={...},            % if you want to delete keywords from the given language
  escapeinside={\%*}{*)},          % if you want to add LaTeX within your code
  extendedchars=true,              % lets you use non-ASCII characters; for 8-bits encodings only, does not work with UTF-8
  frame=single,                    % adds a frame around the code
  keepspaces=true,                 % keeps spaces in text, useful for keeping indentation of code (possibly needs columns=flexible)
  keywordstyle=\color{black},       % keyword style
  morekeywords={*,...},            % if you want to add more keywords to the set
  numbers=left,                    % where to put the line-numbers; possible values are (none, left, right)
  numbersep=5pt,                   % how far the line-numbers are from the code
  numberstyle=\tiny\color{mygray}, % the style that is used for the line-numbers
  rulecolor=\color{black},         % if not set, the frame-color may be changed on line-breaks within not-black text (e.g. comments (green here))
  showspaces=false,                % show spaces everywhere adding particular underscores; it overrides 'showstringspaces'
  showstringspaces=false,          % underline spaces within strings only
  showtabs=false,                  % show tabs within strings adding particular underscores
  stepnumber=1,                    % the step between two line-numbers. If it's 1, each line will be numbered
  stringstyle=\color{mymauve},     % string literal style
  tabsize=2,                       % sets default tabsize to 2 spaces
  title=\lstname                   % show the filename of files included with \lstinputlisting; also try caption instead of title
]
a is:
29 22 28 14 45 10 44 23 9 39 38 52 6 5 50 37 11 26 3 15 2 53 40 54 25 42 12 19 30 16 18 13 1 48 41 24 43 46 47 17 34 20 31 32 33 35 4 49 51 7 36 27 8 21 

call select 1:
find median of medians:
median is in:
28 23 38 15 40 19 18 43 32 35 
call select 2:
find median of medians:
median is in:
28 32 
call select 3:
recursion exit:
the sub-vector whose length is no greater than 5 is:
28 32 
return: 28


left select from recursion exit3
median is: 28

after partition, a becomes:
18 23 19 15 28 40 38 43 32 35 

next, find in left part:
18 23 19 15 28 


left select from left recursion3
call select 4:
recursion exit:
the sub-vector whose length is no greater than 5 is:
18 23 19 15 28 
return: 28


left select from recursion exit4
median is: 28

after partition, a becomes:
21 22 8 14 27 10 7 23 9 4 20 17 6 5 24 1 11 26 3 15 2 13 18 16 25 19 12 28 30 42 54 40 53 48 41 37 43 46 47 50 34 52 31 32 33 35 38 49 51 39 36 44 45 29 

next, find in left part:
21 22 8 14 27 10 7 23 9 4 20 17 6 5 24 1 11 26 3 15 2 13 18 16 25 19 12 28 


left select from left recursion4
call select 5:
find median of medians:
median is in:
21 9 17 11 16 
call select 6:
recursion exit:
the sub-vector whose length is no greater than 5 is:
21 9 17 11 16 
return: 16


left select from recursion exit6
median is: 16

after partition, a becomes:
12 13 8 14 2 10 7 15 9 4 3 11 6 5 1 16 24 26 17 20 23 27 18 21 25 19 22 28 

next, find in left part:
12 13 8 14 2 10 7 15 9 4 3 11 6 5 1 16 


left select from left recursion6
call select 7:
find median of medians:
median is in:
12 9 5 
call select 8:
recursion exit:
the sub-vector whose length is no greater than 5 is:
12 9 5 
return: 9


left select from recursion exit8
median is: 9

after partition, a becomes:
1 5 8 6 2 3 7 4 9 12 15 11 10 14 13 16 

next, find in right part:
12 15 11 10 14 13 16 


left select from right recursion8
call select 9:
find median of medians:
median is in:
12 
call select 10:
recursion exit:
the sub-vector whose length is no greater than 5 is:
12 
return: 12


left select from recursion exit10
median is: 12

after partition, a becomes:
10 11 12 15 14 13 16 

next, find in right part:
15 14 13 16 


left select from right recursion10
call select 11:
recursion exit:
the sub-vector whose length is no greater than 5 is:
15 14 13 16 
return: 15


left select from recursion exit11
the 15th least element of a is: 15

sorted a is:
1 2 3 4 5 6 7 8 9 10 11 12 13 14 15 16 17 18 19 20 21 22 23 24 25 26 27 28 29 30 31 32 33 34 35 36 37 38 39 40 41 42 43 44 45 46 47 48 49 50 51 52 53 54 
the 15th least element of a is: 15
\end{lstlisting}

\end{CJK*}
\end{document}